%%%%%%%%%%%%%%%%%%%%%%%%%%%%%%%%%%%%%%%%%%%%%%%%%%%%%%%%%%%%%%%%%%%%%%%%%%%%%%%%
%%                                                                            %%
%%                                                                            %%
%% An example for writting your thesis using LaTeX                            %%
%% Original version and development work by Luis Costa, changes to the text   %% 
%% in the Finnish template by Perttu Puska.                                   %%https://www.overleaf.com/project/60d1abb1a060481495674645
%% Support for Swedish added 15092014                                         %%
%% PDF/A-b support added on 15092017                                          %%
%% PDF/A-2 support added on 24042018                                          %%
%%                                                                            %%
%% This example consists of the files                                         %%
%%         thesistemplate.tex (version 3.20) (for text in English)            %%
%%         opinnaytepohja.tex (version 3.20) (for text in Finnish)            %%
%%         kandidatarbetsbotten.tex (version 1.00) (for text in Swedish)      %%
%%         aaltothesis.cls (versio 3.20)                                      %%
%%         kuva1.eps (graphics file)                                          %%
%%         kuva2.eps (graphics file)                                          %%
%%         kuva1.jpg (graphics file)                                          %%
%%         kuva2.jpg (graphics file)                                          %%
%%         kuva1.png (graphics file)                                          %%
%%         kuva2.png (graphics file)                                          %%
%%         kuva1.pdf (graphics file)                                          %%
%%         kuva2.pdf (graphics file)                                          %%
%%                                                                            %%
%%                                                                            %%
%% Typeset in Linux either with                                               %%
%% pdflatex: (recommended method)                                             %%
%%             $ pdflatex thesistemplate                                      %%
%%             $ pdflatex thesistemplate                                      %%
%%                                                                            %%
%%   The result is the file thesistemplate.pdf that is PDF/A compliant, if    %%
%%   you have chosen the proper \documenclass options (see comments below)    %%
%%   and your included graphics files have no problems.
%%                                                                            %%
%% Or                                                                         %%
%% latex: (this method is not recommended)                                    %%
%%             $ latex thesistemplate                                         %%
%%             $ latex thesistemplate                                         %%
%%                                                                            %%
%%   The result is the file thesistemplate.dvi, which is converted to ps      %%
%%   format as follows:                                                       %%
%%                                                                            %%
%%             $ dvips thesistemplate -o                                      %%
%%                                                                            %%
%%   and then to pdf as follows:                                              %%
%%                                                                            %%
%%             $ ps2pdf thesistemplate.ps                                     %%
%%                                                                            %%
%%   This pdf file is not PDF/A compliant. You must must make it so using,    %%
%%   e.g., Acrobat Pro or PDF-XChange.                                        %%
%%                                                                            %%
%%                                                                            %%
%% Explanatory comments in this example begin with the characters %%, and     %%
%% changes that the user can make with the character %                        %%
%%                                                                            %%
%%%%%%%%%%%%%%%%%%%%%%%%%%%%%%%%%%%%%%%%%%%%%%%%%%%%%%%%%%%%%%%%%%%%%%%%%%%%%%%%
%%%%%%%%%%%%%%%%%%%%%%%%%%%%%%%%%%%%%%%%%%%%%%%%%%%%%%%%%%%%%%%%%%%%%%%%%%%%%%%%
%%
%% WHAT is PDF/A
%%
%% PDF/A is the ISO-standardized version of the pdf. The standard's goal is to
%% ensure that he file is reproducable even after a long time. PDF/A differs
%% from pdf in that it allows only those pdf features that support long-term
%% archiving of a file. For example, PDF/A requires that all used fonts are
%% embedded in the file, whereas a normal pdf can contain only a link to the
%% fonts in the system of the reader of the file. PDF/A also requires, among
%% other things, data on colour definition and the encryption used.
%% Currently three PDF/A standards exist:
%% PDF/A-1: based on PDF 1.4, standard ISO19005-1, published in 2005.
%%          Includes all the requirements essential for long-term archiving.
%% PDF/A-2: based on PDF 1.7, standard ISO19005-2, published in 2011.
%%          In addition to the above, it supports embedding of OpenType fonts,
%%          transparency in the colour definition and digital signatures.
%% PDF/A-3: based on PDF 1.7, standard ISO19005-3, published in 2012.
%%          Differs from the above only in that it allows embedding of files in
%%          any format (e.g., xml, csv, cad, spreadsheet or wordprocessing
%%          formats) into the pdf file.
%% PDF/A-1 files are not necessarily PDF/A-2 -compatible and PDF/A-2 are not
%% necessarily PDF/A-1 -compatible.
%% All of the above PDF/A standards have two levels:
%% b: (basic) requires that the visual appearance of the document is reliably
%%    reproduceable.
%% a (accessible) in addition to the b-level requirements, specifies how
%%   accessible the pdf file is to assistive software, say, for the physically
%%   impaired.
%% For more details on PDF/A, see, e.g., https://en.wikipedia.org/wiki/PDF/A
%%
%%
%% WHICH PDF/A standard should my thesis conform to?
%%
%% Primarily to the PDF/A-1b standard. All the figures and graphs typically
%% use in thesis work do not require transparency features, a basic '2-D'
%% visualisation suffices. The font to be used are specified in this template
%% and they should not be changed. However, if you have figures where
%% transparency characteristics matter, use the PDF/A-2b standard. Do not use
%% the PDF/A-3b standard for your thesis.
%%
%%
%% WHAT graphics format can I use to produce my PDF/A compliant file?
%%
%% When using pdflatex to compile your work, use jpg, png or pdf files. You may
%% have PDF/A compliance problems with figures in pdf format. Do not use PDF/A
%% compliant graphics files.
%% If you decide to use latex to compile your work, the only acceptable file
%% format for your figure is eps. DO NOT use the ps format for your figures.

%% USE one of these:
%% * the first when using pdflatex, which directly typesets your document in the
%%   chosen pdf/a format and you want to publish your thesis online,

%% * the second when you want to print your thesis to bind it, or
%% * the third when producing a ps file and a pdf/a from it.
%%
\documentclass[english, 12pt, a4paper, sci, utf8, a-1b, online]{aaltothesis}
%\documentclass[english, 12pt, a4paper, elec, utf8, a-1b]{aaltothesis}
%\documentclass[english, 12pt, a4paper, elec, dvips, online]{aaltothesis}

%% Use the following options in the \documentclass macro above:
%% your school: arts, biz, chem, elec, eng, sci
%% the character encoding scheme used by your editor: utf8, latin1
%% thesis language: english, finnish, swedish
%% make an archiveable PDF/A-1b or PDF/A-2b compliant file: a-1b, a-2b
%%                    (with pdflatex, a normal pdf containing metadata is
%%                     produced without the a-*b option)
%% typeset in symmetric layout and blue hypertext for online publication: online
%%            (no option is the default, resulting in a wide margin on the
%%             binding side of the page and black hypertext)
%% two-sided printing: twoside (default is one-sided printing)
%%

\usepackage{graphicx}
\usepackage{listings}

%% Math fonts, symbols, and formatting; these are usually needed
\usepackage{amsfonts,amssymb,amsbsy,amsmath}

%% Change the school field to specify your school if the automatically set name
%% is wrong
% \university{aalto-yliopisto}
% \school{Sähkötekniikan korkeakoulu}

%% Edit to conform to your degree programme
%%
\degreeprogram{Master's Programme in ICT Innovation}
%%

%% Your major
%%
\major{Data Science}
%%

%% Major subject code
%%
\code{SCI3115}
%%
 
%% Choose one of the three below
%%
%\univdegree{BSc}
\univdegree{MSc}
%\univdegree{Lic}
%%

%% Your name (self explanatory...)
%%
\thesisauthor{Cristian Manuel Abrante Dorta}
%%

%% Your thesis title comes here and possibly again together with the Finnish or
%% Swedish abstract. Do not hyphenate the title, and avoid writing too long a
%% title. Should LaTeX typeset a long title unsatisfactorily, you mght have to
%% force a linebreak using the \\ control characters.
%% In this case...
%% Remember, the title should not be hyphenated!
%% A possible "and" in the title should not be the last word in the line, it
%% begins the next line.
%% Specify the title again without the linebreak characters in the optional
%% argument in box brackets. This is done because the title is part of the 
%% metadata in the pdf/a file, and the metadata cannot contain linebreaks.
%%
\thesistitle{Cristian Abrante's amazing thesis}
%\thesistitle[Title of the thesis]{Title of\\ the thesis}
%%

%%
\place{Otaniemi}
%%

%% The date for the bachelor's thesis is the day it is presented
%%
\date{31.12.2020}
%%

%% Thesis supervisor
%% Note the "\" character in the title after the period and before the space
%% and the following character string.
%% This is because the period is not the end of a sentence after which a
%% slightly longer space follows, but what is desired is a regular interword
%% space.
%%
\supervisor{Prof.\ Hong-Linh Truong}
%%

%% Advisor(s)---two at the most---of the thesis. Check with your supervisor how
%% many official advisors you can have.
%%
\advisor{Teemu Sidoroff}
%%

%% Aaltologo: syntax:
%% \uselogo{aaltoRed|aaltoBlue|aaltoYellow|aaltoGray|aaltoGrayScale}{?|!|''}
%% The logo language is set to be the same as the thesis language.
%%
\uselogo{aaltoRed}{''}
%%

%% The English abstract:
%% All the details (name, title, etc.) on the abstract page appear as specified
%% above.
%% Thesis keywords:
%% Note! The keywords are separated using the \spc macro
%%
\keywords{For keywords choose\spc concepts that are\spc central to your\spc thesis}
%%

%% The abstract text. This text is included in the metadata of the pdf file as well
%% as the abstract page.
%%
\thesisabstract{
Your abstract in English. Keep the abstract short. The abstract explains your 
research topic, the methods you have used, and the results you obtained. In the 
PDF/A format of this thesis, in addition to the abstract page, the abstract text is 
written into the pdf file's metadata. Write here the text that goes into the 
metadata. The metadata cannot contain special characters, linebreak or paragraph 
break characters, so these must not be used here. If your abstract does not contain 
special characters and it does not require paragraphs, you may take advantage of 
the abstracttext macro (see the comment below). Otherwise, the metadata abstract 
text must be identical to the text on the abstract page.
}

%% Copyright text. Copyright of a work is with the creator/author of the work
%% regardless of whether the copyright mark is explicitly in the work or not.
%% You may, if you wish, publish your work under a Creative Commons license (see
%% creaticecommons.org), in which case the license text must be visible in the
%% work. Write here the copyright text you want. It is written into the metadata
%% of the pdf file as well.
%% Syntax:
%% \copyrigthtext{metadata text}{text visible on the page}
%% 
%% In the macro below, the text written in the metadata must have a \noexpand
%% macro before the \copyright special character, and macros (\copyright and
%% \year here) must be separated by the \ character (space chacter) from the
%% text that follows. The macros in the argument of the \copyrighttext macro
%% automatically insert the year and the author's name. (Note! \ThesisAuthor is
%% an internal macro of the aaltothesis.cls class file).
%% Of course, the same text could have simply been written as
%% \copyrighttext{Copyright \noexpand\copyright\ 2018 Eddie Engineer}
%% {Copyright \copyright{} 2018 Eddie Engineer}
%%
\copyrighttext{Copyright \noexpand\copyright\ \number\year\ \ThesisAuthor}
{Copyright \copyright{} \number\year{} \ThesisAuthor}

%% You can prevent LaTeX from writing into the xmpdata file (it contains all the 
%% metadata to be written into the pdf file) by setting the writexmpdata switch
%% to 'false'. This allows you to write the metadata in the correct format
%% directly into the file thesistemplate.xmpdata.
%\setboolean{writexmpdatafile}{false}

\colorlet{punct}{red!60!black}
\definecolor{background}{HTML}{EEEEEE}
\definecolor{delim}{RGB}{20,105,176}
\colorlet{numb}{magenta!60!black}

\lstdefinelanguage{json}{
    basicstyle=\normalfont\ttfamily,
    numbers=left,
    numberstyle=\scriptsize,
    stepnumber=1,
    numbersep=8pt,
    showstringspaces=false,
    breaklines=true,
    frame=lines,
    backgroundcolor=\color{background},
    literate=
     *{0}{{{\color{numb}0}}}{1}
      {1}{{{\color{numb}1}}}{1}
      {2}{{{\color{numb}2}}}{1}
      {3}{{{\color{numb}3}}}{1}
      {4}{{{\color{numb}4}}}{1}
      {5}{{{\color{numb}5}}}{1}
      {6}{{{\color{numb}6}}}{1}
      {7}{{{\color{numb}7}}}{1}
      {8}{{{\color{numb}8}}}{1}
      {9}{{{\color{numb}9}}}{1}
      {:}{{{\color{punct}{:}}}}{1}
      {,}{{{\color{punct}{,}}}}{1}
      {\{}{{{\color{delim}{\{}}}}{1}
      {\}}{{{\color{delim}{\}}}}}{1}
      {[}{{{\color{delim}{[}}}}{1}
      {]}{{{\color{delim}{]}}}}{1},
}

%% All that is printed on paper starts here
%%
\begin{document}

%% Create the coverpage
%%
\makecoverpage

%% Typeset the copyright text.
%% If you wish, you may leave out the copyright text from the human-readable
%% page of the pdf file. This may seem like a attractive idea for the printed
%% document especially if "Copyright (c) yyyy Eddie Engineer" is the only text
%% on the page. However, the recommendation is to print this copyright text.
%%
\makecopyrightpage

%% Note that when writting your thesis in English, place the English abstract
%% first followed by the possible Finnish or Swedish abstract.

%% Abstract text
%% All the details (name, title, etc.) on the abstract page appear as specified
%% above.
%%
\begin{abstractpage}[english]
  Your abstract in English. Keep the abstract short. The abstract explains your
  research topic, the methods you have used, and the results you obtained.  
  
  The abstract text of this thesis is written on the readable abstract page as
  well as into the pdf file's metadata via the $\backslash$thesisabstract macro
  (see above). Write here the text that goes onto the readable abstract page.
  You can have special characters, linebreaks, and paragraphs here. Otherwise,
  this abstract text must be identical to the metadata abstract text.
  
  If your abstract does not contain special characters and it does not require
  paragraphs, you may take advantage of the abstracttext macro (see the comment
  below).
\end{abstractpage}

%% The text in the \thesisabstract macro is stored in the macro \abstractext, so
%% you can use the text metadata abstract directly as follows:
%%
%\begin{abstractpage}[english]
%	\abstracttext{}
%\end{abstractpage}

%% Force a new page so that the possible Finnish or Swedish abstract does not
%% begin on the same page
%%
\newpage

%% Preface
%%
%% This section is optional. Remove it if you do not want a preface.
\mysection{Preface}

I want to thank Professor Pirjo Professori and my instructor Dr Alan Advisor for 
their good and poor guidance.\\

\vspace{5cm}
Otaniemi, 31.12.2020

\vspace{5mm}
{\hfill Cristian M.\ Abrante Dorta \hspace{1cm}}

%% Force a new page after the preface
%%
\newpage


%% Table of contents. 
%%
\thesistableofcontents


%% Symbols and abbreviations
\mysection{Symbols and abbreviations}

\subsection*{Symbols}

\begin{tabular}{ll}
$\mathbf{B}$  & magnetic flux density  \\
$c$              & speed of light in vacuum $\approx 3\times10^8$ [m/s]\\
$\omega_{\mathrm{D}}$    & Debye frequency \\
$\omega_{\mathrm{latt}}$ & average phonon frequency of lattice \\
$\uparrow$       & electron spin direction up\\
$\downarrow$     & electron spin direction down
\end{tabular}

\subsection*{Operators}

\begin{tabular}{ll}
$\nabla \times \mathbf{A}$              & curl of vectorin $\mathbf{A}$\\
$\displaystyle\frac{\mbox{d}}{\mbox{d} t}$ & derivative with respect to 
variable $t$\\[3mm]
$\displaystyle\frac{\partial}{\partial t}$  & partial derivative with respect 
to variable $t$ \\[3mm]
$\sum_i $                       & sum over index $i$\\
$\mathbf{A} \cdot \mathbf{B}$    & dot product of vectors $\mathbf{A}$ and 
$\mathbf{B}$
\end{tabular}

\subsection*{Abbreviations}

\begin{tabular}{ll}
SOA          & Service oriented architecture \\
DOMA         & Domain-oriented microservices architecture \\
WSDL         & Web Service Deskcription Language \\
SOAP         & Simple Object Acess Protocol \\
RPC          & Remote Procedure Call \\

\end{tabular}


%% \clearpage is similar to \newpage, but it also flushes the floats (figures
%% and tables).
%%
\cleardoublepage

%% Text body begins. Note that since the text body is mostly in Finnish the
%% majority of comments are also in Finnish after this point. There is no point
%% in explaining Finnish-language specific thesis conventions in English.
%% This text will be translated to English soon.
%%

%%
%% Introduction.
%% In this part I have to state why the thesis is relevant.
%% Just check the thesis proposal.
%% 
\section{Introduction}

%% For the introduction I guess it can be mostly copied from the thesis proposal,
%% anyway one of the last sections to write.

\subsection{Contributions}

% Explain the contributions introduced in that thesis.

\subsection{Structure of the thesis}

% Easy section, it can be mostly written now.

%% Leave page number of the first page empty
\thispagestyle{empty}

\clearpage
\section{Background}
\label{sec:background}

%% The objective of the background section is to give a business context to explain where the problem come from.

This section describes all the necessary background to understand the problem that this thesis seeks to address. In the first subsection, a brief introduction is made about the company and the business environment in which it operates, and in the second subsection, the problem is framed from different points of view, taking into account the various stakeholders involved.

\subsection{Company description}

%% Describe Unity technologies as a company.
%% Describe Unity Acquire architecture.
%% Describe Unity Acquire Microservices architecture.
%% Describe Unity Services gateway and why it is needed.

To better understand the project, it is essential to present the company where it has been developed. This thesis is done in collaboration with Unity Technologies \cite{UnityTechnologies}, a multinational company based in the United States. The company's main product is the Unity game engine \cite{haas2014history}, an industry standard for game and 3D content development  \cite{nicoll2019unity}.\\

Although the game engine is the most famous product offered by the company, it is not the only source of revenue, and it is not enough to understand its business model. This is mainly because the company's value proposition can be summarized as: \textit{"Create a platform for creators to develop, execute and monetize their content"} \cite{UnityBusinessModel}. To realize this value proposition, the company has established two organizational departments that offer various products to its users.\\

\begin{figure}[h]
    \centering
    \includegraphics[scale=0.3]{src/thesis/img/background/unity-division.jpg}
    \caption{Products offered in the two organizational divisions of Unity Technologies}
    \label{fig:unity-solutions}
\end{figure}

\begin{itemize}
    \item \textbf{Create solutions}: The main functionality of this product suite is to offer a set of tools that allows users to produce all kinds of 3D content. The business model of this division consists of a subscription-based model for the different versions of the Unity Editor.
    \item \textbf{Operate solutions}: The goal of the products categorized in this section is to enable creators to run a successful business from their 3D content. This is the organizational section that contains UnityAds, the product on which this thesis project was developed.
\end{itemize}

\subsubsection{UnityAds}

UnityAds is one of the most outstanding products offered by the company. To understand it correctly, it is essential to clarify the concept of \textbf{online advertising network} \cite{goldfarb2011online} \cite{schmeiser2016online}. \\

An advertising network is a type of business or service whose purpose is to connect advertisers, who are individuals who want to display advertisements for the products they intend to sell, with publishers, who are individuals who can provide space to place such advertisements, and who usually reach a certain level of audience. The original definition of advertisement network was really tied to newspapers, magazines, and television, but in recent years, it has been extended to online services \cite{ha2008online}. In that sense, UnityAds propose a new innovative business idea because it is an ad-hoc advertisement network for the game industry. In a nutshell, UnityAds allows game developers to show ads about their games in other games and place advertisements of other games in theirs \cite{chu2013iad}. As can be seen, by using UnityAds, the same game developer can be publisher and advertiser simultaneously, melting both roles.\\

\begin{figure}[h]
    \centering
    \includegraphics[scale=0.4]{src/thesis/img/background/unity-ads.png}
    \caption{Simplified structure of UnityAds}
    \label{fig:background:unity-ads}
\end{figure}

Figure \ref{fig:background:unity-ads} shows the two aspects that make up the advertising network: \textbf{Acquire} and \textbf{Monetize}. The terms correspond to the classic definition of publisher and advertiser but translated into the gaming industry terminology.

\begin{itemize}
    \item \textbf{Monetize division}: This section includes all the tools provided to game developers to monetize their games, in addition to all the technical solutions needed to deliver an ad to the mobile game that has a publisher space.
    \item \textbf{Acquire division}: This set of tools includes all the products that help advertisers acquire new users who can install the games they advertise on the network.
\end{itemize}

Although this overall context of UnityAds was essential to understand the big picture, only the Acquire division is considered in the following sections, as the thesis project focuses on it.

\subsubsection{Acquire architecture overview}
\label{sec:acquire-architecture}

This section covers the architecture overview for the acquire division of UnityAds. It is important to note that many details of the architecture will not be explained in-depth to respect the company's privacy.\\

The architecture of UnityAds has changed dramatically over the past decade. Initially, the ad network was developed by a Finnish startup, Applifier, acquired by Unity Technologies in 2014 \cite{UnityBuyApplifier}. The initial architecture of the Applifier ad network can be seen in figure \ref{fig:background:applifier-architecture}.

\begin{figure}[h]
    \centering
    \includegraphics[scale=0.4]{src/thesis/img/background/applifier-unityads-architecture.png}
    \caption{Diagram for the initial architecture of Applifier}
    \label{fig:background:applifier-architecture}
\end{figure}

This almost \textit{monolithic architecture} \cite{de2019monolithic} impacted the startup times, as it was convenient for a small engineering team to introduce changes to it, which allowed them to achieve a solid customer base prior to the Unity acquisition \cite{ApplifierCustomers}. But when scalability started to become a fundamental requirement, there were some pain points they had to address. On the one hand, the database was the junction point between the services and the mechanism they used to communicate, which was limiting and error-prone, and on the other hand, this architecture was not suitable to be managed by a growing team of engineers. Due to this, the company started to migrate its setup into a microservices-oriented architecture (Section CITE-SECTION). This means that now the database is not the single point of communication, and each of the services would need an independent database to store the data they needed to operate. The acquire and monetize division can easily be decoupled using that paradigm, and different teams could work independently. This initiative led the company to its current situation, which can be observed in figure \ref{fig:background:acquire-architecture}.\\

\begin{figure}[h]
    \centering
    \includegraphics[scale=0.35]{src/thesis/img/background/acquire-division.png}
    \caption{Current architecture of Acquire division of UnityAds}
    \label{fig:background:acquire-architecture}
\end{figure}

On this new setup, Monetization and Acquire can be considered domains on the domain-oriented service architecture of UnityAds (Section CITE-SECTION). It is also important to mention that the communication between microservices uses different protocols; for example, for the public-facing endpoints, the connection is established using HTTPS \cite{durumeric2013analysis}, and for the internal communication between the microservices in the domain, NATS \cite{srisuresh2008state} is the selected protocol, finally, between the monetization and the acquire domains, is Kafka \cite{kreps2011kafka} the chosen technology.\\

One noticeable thing on the diagram is the existence of two gateways as the entry point for the public-facing endpoints. Some of the calls, like the ones provided from the client dashboard, are sent through the legacy gateway, whereas some requests, like those coming from the public APIs, use Unity Services Gateway. The purpose for the existence of two gateways can be traced back some years, while the planning phase of the current domain-oriented service architecture.\\

Before the acquisition of Applifier by Unity, an initial gateway compiled some simple high-level responsibilities (Section CITE-SECTION) that were refactored to add more functionalities like rate limiting, authentication, authorization, and error handling. The legacy gateway is highly tied to the monetization environment, making it difficult to add new functionalities or address pre-existing problems. This is why a new initiative consists of creating a new project, the Unity Services Gateway, whose objective is to serve as the entry point for all the services both on the acquire and monetization domains and possibly across all the company.\\

\subsubsection{Microservices monitoring}

The last step to understand UnityAds architecture is to explain how microservice monitoring is done (Section CITE-SECTION). It is essential to state that all the architecture is deployed on Google Cloud \cite{aceto2013cloud}, which already offers some pre-defined tools that make life easier to operate the product.\\

\begin{figure}[h]
    \centering
    \includegraphics[scale=0.4]{src/thesis/img/background/service-monitoring.png}
    \caption{UnityAds microservices monitoring diagram}
    \label{fig:background:monitoring}
\end{figure}

On figure \ref{fig:background:monitoring} it can be observed a diagram that represents the selected monitoring technologies for the Acquire architecture. As stated before, this architecture is running on a Kubernetes cluster that is deployed on Google Cloud \cite{hunter2018google}. That makes it easier to use log sinks to transfer the logs to the standard google cloud logging tool. But logging is only one of the relevant parts for monitoring, as metrics are also essential to know the overall system's health.\\

In order to collect the metrics for the different microservices of the system, Prometheus \cite{turnbull2018monitoring} is the selected tool. It works by pulling relevant metrics from microservices and infrastructure, and afterward, those metrics can be accessed using its custom query language. On top of it, some Grafana dashboards \cite{hoang2020research} can be handcrafted to visualize those metrics on time series plots. This typical setup works well as it combines the possibility of exploring error logs and even tracing the endpoint calls for the microservices on a fine granularity, using standard Google Cloud tools such as the profiler \cite{thakurratan2018google}, and at the same time presents a visual representation for the metrics of the system, such as error rate, uptime or throughput, using a time-series format.\\

\subsection{Problem definition}
\label{sec:problem-definition}

%% Describe the problem from different angles, depending on the stakeholders involved.

%% - Public APIs team: Describe that from our side the requirements is having more
%%                     visibility in the term of the errors that are generating through
%%                     our API calls from the engineering part of view.

%% - Product management: Describe that from the product side, it is important to know the high
%%                       level errors as they are in contact with the client, and they have to 
%%                       contact them in case that they are doing some wrong integrations, for example
%%                       if they hit the rate limits.

%% - Security team: From their perspective it is important to set some visibility on the high level errors
%%                  in case that some individual malicious user is doing some kind of attack to the system.

Even though the monitoring setup adopted by the Acquire unit on UnityAds works well in many cases, it has some limitations related to the monitoring of the Unity Services Gateway. First of all, it is remarkable to say that Unity Services Gateway is another microservice, so the setup represented in figure \ref{fig:background:monitoring} is enough for metrics and architecture monitoring. Nevertheless, it has some limitations for business logic monitoring \cite{von2009decision}.\\

A gateway of a Domain-Oriented microservices architecture is in charge of handling some high-level responsibilities (Section CITE-SECTION). The main advantage is that those tasks are shared across microservices, so there is a single place where this logic is implemented and maintained. But also, the teams responsible for the microservices that are directly dependant on the gateway might want some visibility on the events generated by handling that logic.\\

To give a concrete example for this problem, let's consider only rate-limiting responsibility \cite{raghavan2007cloud}, which is vital to prevent common attacks \cite{jhaveri2012attacks}. The gateway's commitment corresponds to automatically rejecting the requests that had tried to access the server multiple times. To do that, it has to keep a temporal database that stores the count for how many requests the user had done on a concrete time window.\\

Considering this error type, different stakeholders would be interested in knowing more about the events associated with this error and that currently don't have that much information.

\subsubsection*{Public APIs team}

This team is responsible for developing a set of public APIs that allow advertisers to manage their advertisement campaigns programmatically. In Figure \ref{fig:background:acquire-architecture}, it can be seen that the service owned by this team serves as a proxy between the backends and the API client. Usually, those clients are third-party advertisers that have built custom integrations with UnityAds and want to run adverts on different platforms at the same time to maximize the visibility of their campaigns.\\

The engineering team is contacted when some advertiser is constantly receiving errors from the APIs because it has reached the established rate limit for the endpoint they are trying to consume. This can happen for a variety of reasons, maybe they want to fetch a resource that needs multiple calls to be completed, or there is an error in the implementation that causes the endpoint to be called enough times to hit the rate limit. With the current setup, it is tough to trace back the reasons as even if the engineering team can access the logs, as they might not contain all the information in a convenient format. Another drawback is that the engineering team does not notice the problem beforehand, but only after it happens and the clients have already contacted them.

\subsubsection*{Product managers and client partners}

Product managers and client partners are the intermediaries between the engineering team and the clients that consume the APIs or any other connected service through the gateway. This is also why they want to get some visibility in which clients are hitting the rate limits on the gateway level. If they could identify if any client is hitting the rate limits, they can contact them directly to ensure that their integration or their usage of the company's services is correct and give assistance in that regard if needed. In that way, customer satisfaction could be improved with respect to the current status, where is the client who usually contacts if they are experimenting many errors.\\

In that sense, this team is interested in having some visual tool to access that information because accessing the logs console using a series of queries requires some previous background that those professionals might not have.

\subsubsection*{Security team}

The security team wants to know more when specific clients or users make several requests to the services. Because in case that it is not an already identified client, which can be directly contacted, but another malicious user, they can ban him to prevent any further attack \cite{jhaveri2012attacks}.

\clearpage
\section{Requirements engineering}

%% Taking into consideration that everyone has clear who are the stakeholders and what is the structure of the
%% company and the microservices involved, now we have to do the requirement analysis.

%% First of all, it is important to state why it is not possible to store things on Prometheus.

%% We can list the project requirements like:

%% - We need to do a clear definition of the errors that we want to log.
%% - The logging technology has to support a lot of stress as we are having x (collect the number of average requests).
%% - The provided solution has to be cloud agnostic (this is why some part has to be implemented in Terraform or using an external service).
%% - We need a visual tool in order to communicate effectively with the stakeholders.

Having presented the background necessary to understand the thesis project in Section \ref{sec:background}, this section will cover requirements engineering.\\

Requirements engineering is the process of establishing the desired requirements to be met by the project \cite{hay2003requirements}. Section \ref{sec:req-methodology} will define the methodology to be followed to gather the requirements, and Section \ref{sec:req-analysis} will analyze these requirements. Finally, Section \ref{sec:research-questions} will present the research questions that will serve as the guiding thread of this thesis.

\subsection{Requirements engineering methodology}
\label{sec:req-methodology}

Before performing the requirements analysis, it is essential to clarify which methodology will be followed to identify the project requirements. In this case, the framework proposed by David C. Hay \cite{hay2003requirements} for requirements engineering, which consists of three steps, is selected:

\begin{itemize}
    \item \textbf{Eliciting requirements}: this phase is where all the requirements will be discovered following the selected gathering methodology.
    \item \textbf{Recording requirements}: The second phase consists of the documentation of the requirements previously collected.
    \item \textbf{Analyzing requirements}: Finally, those requirements have to be analyzed in order to see if they are clear, complete, concise, and unambiguous.
\end{itemize}

For the elicitation of requirements, the selected method is stakeholders identification \cite{mitchell1997toward}, which consists of selecting individuals or groups interested in the product or function to be developed and asking them what the desired characteristics they expect from the product are. This method is the most appropriate for our project since it will only have an internal impact as external customers will not use it. The list of stakeholders for this project is given in section \ref{sec:problem-definition}.\\

To record and document the requirements expressed by the stakeholders, there were two methods: the first was to use message threads in Slack, which is the internal communication platform, where relevant stakeholders could set their expectations, and the second method was to organize some meetings, based on the framework of the Joint Requirements Development (JRD) Sessions \cite{hay2003requirements}, to discuss the project and the expected results. At these meetings, a log was drafted to keep track of the issues discussed.

\subsection{Requirements analysis}
\label{sec:req-analysis}

The last step of the chosen requirements engineering framework consists of performing an analysis of the requirements gathered from stakeholders. In that sense, the solution that is going to be presented for solving the problem stated in section \ref{sec:problem-definition} needs to meet the following requirements:

\begin{itemize}
    \item Clear definition on which are going to be the high-level responsibilities that the gateway will handle and which errors the gateway can generate associated with those.
    \item Establish a format for the generated errors. They need to store all necessary fields in order to be traced back and visualized afterward.
    \item Map possible errors and gateway responsibilities to the different endpoints that clients can consume.
    \item Select a suitable data store for the errors. Data storage needs to be easily integrated with the current cloud setup and must ensure the acceptance of a standard query language that allows the integration with third-party tools.
    \item Data ingestion has to be cloud-agnostic. This means that the error logging pipeline, from the event emission to the storage, has to be done with a technology that ensures compatibility with multiple cloud providers.
    \item Develop a visualization tool that allows stakeholders to access the errors generated at the gateway level easily.
\end{itemize}

\subsection{Research questions}
\label{sec:research-questions}

%% Taking into consideration the previous requirement analysis we can state down the research questions.

This project aims not only to satisfy the specific need for a technical feature but also to generate a set of tools and practices that can be used in the future by any other organization operating a DOMA. Therefore, this project needs an extensive investigation and planning phase, which these three research questions will guide:

\begin{itemize}
    \item[\textbf{Q1}] \textit{What is the best way to represent the possible high-level errors that a DOMA gateway can produce?}
    \item[\textbf{Q2}] \textit{What is the optimal technology for storing the errors that a DOMA gateway can produce?}
    \item[\textbf{Q3}] \textit{How to correctly visualize and communicate with the stakeholders the errors a DOMA gateway can produce?}
\end{itemize}

The objective of the thesis is to produce a series of software artifacts and standards that will satisfy the research questions and, therefore, the technical requirements of the stakeholders.

\clearpage
\section{Literature review}

%% In this section is where all the technologies used are reviewed. For sure having the following subsections:

This section aims to provide a broad review of some topics relevant to the development of this thesis. The first subsection provides a review of today's essential service architectures, including the new DOMA architecture. The second subsection provides a review of observability and monitoring issues.

\subsection{Software architectures review}

%% 5 pages
%% In this section we can include the following: (5 pages)
%% - Brief introduction about the theoretical aspects that are going to be discussed.
%% - Service Oriented Architectures (SOA)
%% - Microservces architecture.
%% - Domain oriented microservices architecture. Maybe introduce the gateway here or in another page.

Domain Oriented Microservices Architecture (DOMA) is the paradigm that many medium and large organizations are adopting to organize their software systems and the most suitable to describe the software architecture of the Acquire unit (Section  \ref{sec:acquire-architecture}). This section offers a review of the software architectures that led to the establishment of DOMA as the state-of-the-art.\\

Software architecture is a fundamental topic in software engineering. Its main objective is to investigate the optimal way to organize software processes and artifacts to ensure that they can function properly and communicate with each other to accomplish their tasks \cite{shadija2017towards}. According to E. Perry and L. Wolf \cite{perry1992foundations}, the term \textit{"architecture"} was introduced into software engineering in the 1980s to evoke the notions of abstraction and coding. It is still relevant today to define the layout of software systems in different enterprises.\\

One of the initial paradigms that established a breakthrough in how software systems were organized was the introduction of object-oriented programming (OOP) \cite{dahl1972chapter}, which was the first attempt to encapsulate data, offering a limited set of operations to access and modify it. It also introduced the concept of interfaces \cite{snyder1993essence}, which are the entry point that agents external to the object had to access. Although OOP established a solid foundation for creating more complex systems, it was still not sufficient to map business functionalities, which were later satisfied using more complex software paradigms. In the following subsections, those complex paradigms that are capable of mapping business functionality directly will be explained.

\subsubsection{Service-oriented architecture}

Service-Oriented Architecture (SOA) is a software paradigm that consists of packaging software systems into services that can be presented to the user in condensed packages that can be accessed through a well-defined protocol \cite{sprott2004understanding}. The rise in popularity of SOA coincides with the establishment of web protocols that allow software systems to communicate effectively.\\

The usage of those services can be done in two different ways. On the one hand, services can be consumed at run time by binding them into the code; thus, the packaging of the services is done in software libraries \cite{alahmari2010service}. On the other hand, those services can be consumed using some web protocol; in this case, some standards can be used to establish a data format, such as Web Service Description Language (WSDL) or Simple Object Access Protocol (SOAP).\\

Even though SOA is a giant step forward in the design of software systems, mainly because it helps on the scalability and reusability of components effectively, it can also present some drawbacks:

\begin{itemize}
    \item Services can contain many functionalities and a big codebase, which can be challenging to manage if the engineering team is constantly growing.
    \item If the business domain requires many functionalities that can be grouped logically in the same service, it can rapidly become a monolith. 
    \item Adding new functionalities to the system usually implies that the whole service has to be deployed simultaneously, which can be error-prone.
    \item Data storage is a central piece of software systems. It can also be the communication point between services, which can cause inconsistencies as several services might get access to the same data.
\end{itemize}

\begin{figure}[h]
    \centering
    \includegraphics[scale=0.4]{src/thesis/img/literature-review/SOA.png}
    \caption{Diagram describing SOA}
    \label{fig:soa-architecture}
\end{figure}

This software architecture might be helpful in the initial stage of companies or organizations, where services are minimal and the functionalities are not entirely developed. But the pain points might arise when operating on a different scale. The initial architecture of Applifier can be considered as SOA (Figure \ref{fig:background:applifier-architecture}), as it shared the same structure and had to deal with similar drawbacks.

\subsubsection{Microservices architecture}
\label{sec:microservices-architecture}

Microservices architecture is a novel approach that has gained a lot of interest both from the industrial and academic perspectives to solve the problems that traditional service architecture presents.\\

Many authors have tried to give precise definitions for microservices to establish a framework that can help both engineers and researchers develop applications that could tackle the original problems of software services.\\

One of the most relevant definitions is done by Dragoni et al. \cite{dragoni2017microservices}: \textit{"A microservice is a cohesive, independent process interacting via messages."} This definition is quite interesting because it is not centered on the size of the service, which is one of the debate points of the description due to the introduction of the word \textit{"micro"} on it. It also emphasizes the concept of independence and remote communication, one of the critical points of modern microservices, which are usually deployed independently and communicate with each other using a well-known protocol.\\

In that regard, the definition of Johannes Thönes certainly expands some relevant aspects \cite{thones2015microservices}: \textit{"Microservice is a small application that can be deployed independently, scaled independently, and tested independently, and that has a single responsibility."} One of the emphasis points on this definition is the independence on the deployment and testing, which is one of the key advantages that make microservices useful when managed by a growing team of engineers. Apart from that, Thrones points out the single responsibility principle that should always be kept in mind when designing microservices.\\

Following the classification of Lewis et al. \cite{MicroservicesCharacteristics} microservices should share the following characteristics:

\begin{itemize}
    \item \textbf{The modularity of services}. As in SOA, microservices offer a modular functionality that can be reused easily. The main difference between both is that microservices could be deployed independently, whereas, on SOA, each release implies a complete deployment of the service.
    \item \textbf{Organised around business capability}. On traditional software systems, teams are organized into different functional areas; for example, one group would be responsible for engineering one service and another for deployment and testing. In that schema, if there is a failing deployment, the responsibility is shared across the teams, which can cause some delays as the issue can not be handled within an organizational unit.
    \item \textbf{They are product not projects}. In SOA, each service is mapped into a business requirement; on the contrary, on microservices, one business requirement aims to be mapped to multiple microservices to reduce the amount of responsibilities of each one.
    \item \textbf{There are intelligent endpoints and dumb pipes}: The main logic has to be placed on the microservices that are accessed via endpoints. There have to be few communication middlewares (pipes) with actual logic.
    \item \textbf{Decentralized data management and governance}: One of the main characteristics of microservices is that data is no longer the common point between services because each microservice has its independent data store.
\end{itemize}

\begin{figure}[h]
    \centering
    \includegraphics[scale=0.4]{src/thesis/img/literature-review/microservices.png}
    \caption{Diagram describing microservices architecture}
    \label{fig:microservices-architecture}
\end{figure}

Microservices represent a giant step forward in developing software architectures, as they present many \textit{operational} benefits compared to SOA. Nevertheless, they are not a silver bullet as it also introduces some significant drawbacks, such as:

\begin{itemize}
    \item The introduction of performance issues. The remote procedure calls (RPC) used between microservices to communicate can introduce some delays as they have to be resolved over the network.
    \item Problems on the data aggregation.  As microservices remove a shared data storage in favor of a decoupled data governance if some complex data has to be retrieved that involves some integration between different sources.
    \item Difficulties in introducing new features. In a microservice architecture, introducing a new feature might involve changing several services, which can delay communication, validation, and code review.
\end{itemize}

\subsubsection{Domain-oriented microservices architecture}

% The schema on this part has to be as it follows.
% - Introducing the main problems that microservices architecture present for large scale companies.
% - The different parts of the microservices architecture: domains, layers and gateways.
% - Review of Unity´s initial DOMA architecture.

The latest revised architecture will be DOMA, presented by Gluck A. in 2020, on behalf of Uber's engineering team \cite{DOMAUber}. The primary goal of DOMA is to mitigate the problems that microservices architecture presents in large enterprises that have hundreds of interconnected microservices.\\

In these companies, there are some additional problems to those discussed in section \ref{sec:microservices-architecture}:

\begin{itemize}
    \item Knowledge sharing and architecture onboarding. When the number of microservices is significant, the understanding of the system becomes complex. That increases the onboarding time for new engineers, which causes an operational cost for the company.
    \item Difficulties to to debug and identify a problem. When considering hundreds of interconnected services, if a call in any of them introduces some delay or bug, engineers have to trace back the call stack, requiring substantial cross-team effort.
\end{itemize}

DOMA is a new software design pattern iteration that tries to tackle these problems. The main principles are extracted from Domain-Driven Design, a paradigm introduced by E. Evans and EJ. Evans \cite{evans2004domain}, which defines software design based on the domain or business-specific needs. In that regard, DOMA is the direct application of Domain-Driven design to microservices design in an extensive network of distributed and interconnected services.\\

\begin{figure}[h!]
    \centering
    \includegraphics[scale=0.3]{src/thesis/img/literature-review/doma.png}
    \caption{DOMA architecture diagram, representing the possible }
    \label{fig:my_label}
\end{figure}

DOMA is adopted on organizations as an iteration of a pre-existent microservices software architecture. Four main elements have to be introduced \cite{DOMAUber}, to do that transition:

\begin{itemize}
    \item \textbf{Domains}: A domain is a collection of related microservices. The main design challenge of this element is to decide how big the domain should be and how many microservices should contain.
    \item \textbf{Layers}: A layer is a collection of domains. The primary behavioral change is that we can consider that the layers are grouped into a stack, where the layer defines the dependencies between the domains that they belong to in the pyramid.
    \item \textbf{Gateway}: The gateway is the single point of access of each domain. It is the public interface of the domain, and it should prevent access to individual services and implement several high-level responsibilities.
    \item \textbf{Extension architecture}: the extension mechanism is a procedure for adding extra functionalities to the ones already present on the domain. This is important to respect the open-closed principle of the domain. It is important to remark that this can not be achieved on all domains as it depends on the nature of the domain's data.
\end{itemize}

The introduction of DOMA is recent, so there is not much research on how to populate those four elements with concrete business requirements to transition a microservices architecture into a successful DOMA. The only available public example is offered in the article by Uber Engineering \cite{DOMAUber}, where they provide an interesting view on how to distribute the layers pyramidally, where the base of the pyramid represents broad functionalities that can be used across the organization, such as storage or networking services. In contrast, concrete client-facing functionalities are at the top of the pyramid. This interesting approach guarantees that the layers are isolated, and dependencies flow from top to bottom.

\subsection{Microservices gateway}
\label{sec:microservices-gateway}

This section is intended to extend the knowledge presented in the previous section about microservices gateways. As stated before, gateways are an essential element in DOMA, and their primary purpose is to isolate the domains of microservices and work as the single entry point. Even though they represent a crucial element in DOMA, they can also be used in a simple microservices architecture.\\

\begin{figure}[h!]
    \centering
    \includegraphics[scale=0.3]{src/thesis/img/literature-review/gateway.png}
    \caption{Diagram comparing the call of a microservice architecture with or without gateway}
\end{figure}

On Microservices Patterns, by C. Richardson \cite{richardson2018microservices}, several reasons why creating an API gateway is vital on a microservices architecture are introduced.

\begin{itemize}
    \item Each microservice exposes a minimal set of fine-grain API endpoints. To compose a more meaningful and high-level API call, the information of many microservices has to be integrated somehow. That responsibility should ideally not lie on the consumers of the services.
    \item The users do not need to know the internal details of microservices architecture to respect the company's privacy and internal structure. Due to this, they rather interact with some proxy instead of calling the specific microservices.
    \item The communication protocols between microservices can be very diverse. Clients would prefer only to use one communication protocol to request resources on the systems rather than preparing their custom data integration.
    \item Several high-level responsibilities are shared between microservices, and they would be better managed on a central point rather than re-implementing the logic for each service.
\end{itemize}

Each point of this list defines a valid reason for considering the creation of a microservices architecture gateway; nevertheless, the last one is the most relevant for this thesis and is the one that will be explored in-depth.\\

As this thesis aims to monitor the high-level errors of a microservices architecture gateway, it is necessary first to define the responsibilities it has to take. The high-level responsibilities stated by M. Thangavelu et al. \cite{UberGateway} will be considered for this definition.

\begin{itemize}
    \item \textbf{Auditing pipeline}. maintaining a registry of all the access and operations that the gateway performs. This thesis aims to improve the logging when referring to the error generated by the stated responsibilities.
    \item \textbf{Identity}. Each access to the service has to be authenticated and authorized, and that responsibility should lie on a central software piece like the gateway.
    \item \textbf{Rate limiting}. That refers to the capacity of the system to prevent malicious attacks caused by an uncontrolled number of recurrent requests.
    \item \textbf{Documentation}. Ideally, all API calls should be documented on a central repository with a standard format.
    \item \textbf{Response fields trimming}. As the gateway works as a proxy for the backend microservices, it would be helpful to have the possibility to specify the omission of specific fields on the user's responses.
    \item \textbf{Datacenter affinity}. There can be some logic on the gateway that could redirect user calls to certain physical regions where the data can be retrieved faster.
    \item \textbf{Short-term user bans}. Temporarily banning users who had malicious behavior on the system can be one of the solutions to deal with cyber-attacks. That responsibility should be delegated to a central software piece rather than be shared across the microservices.
\end{itemize}

Considering those high-level responsibilities, in section CITE-SECTION the errors related to each of them will be defined.

\subsection{Observability \& Monitoring}

%% In this section we can differentiate the following topics: (5 pages)
%% - Observability and Monitoring definition.
%% - Application of observability and monitoring to microservices.
%% - Data observability.
%% - The four pillars of data observability (check one of the articles).

\subsection{Framework comparisons}

%% (10 pages)
%% In this section a comparison of the current frameworks for monitoring and observability has to be done.
%% The following can be compared:

%% - Google custom monitoring & observability tools.
%% - Datadog
%% - Amplitude

%% It is important to explain the features of all of them when monitoring microservices, also taking into consideration
%% the costs analysis, and if they are suitable for showing the custom data that we are going to log.

\clearpage
\section{Technical solution}

COMPLETE-SECTION -> ON THIS PART GIVE A BRIEF INTRODUCTION ON WHICH ARE THE TECHNICAL SOLUTIONS DEVELOPED.

%% The outcome of this thesis are basically 4 technical solutions.

\subsection{Gateway configuration file}

After having discussed the high-level responsibilities that the gateway of a DOMA should have (section \ref{sec:microservices-gateway}), in this chapter, one of the main outputs obtained from this thesis is introduced, which is the definition of a route-level configuration format for the gateway, in order to solve \textbf{Q1}.\\

A route-level configuration format is a standard that can be used to state down all the resources that the gateway of the domain exposes to the public and the associated responsibilities attached to each route. Having this standard defined will have many significant advantages, such as:

\begin{itemize}
    \item It can be used as the primary source of truth of the resources that the gateway exposes.
    \item The generation of the endpoint's documentation can be automatized; therefore, all the teams involved in the microservices do not need to do it by themselves.
    \item This standard defines a contract on what the endpoint handlers should do when managing the responsibilities. The engineers that are going to implement the proxy should fulfill the specifications of the standards, and the endpoints consumer should rely on those responsibilities.
    \item There are some literature examples of composable gateway logic \cite{GatewayUber}; this standard can be the consuming point of this automated gateway generator. 
\end{itemize}

JSON schema is the standard that will be used for defining the gateway configuration file. The primary purpose of using that schema is that several tools can be used to validate the configuration file against the schema, such as \texttt{ajv} \cite{AjvTool} or \texttt{is-my-json-valid} \cite{IsMyJsonValid}. The schema is defined using JSON format, and it is publicly available on the open-source repository for this project (CITE-SECTION).\\

Even though the schema is written in a JSON format, we can expect to have either JSON or YAML for the actual configuration file, as it is more human-readable. With the independence of the format used, the file should match the specified validation.\\

In the following subsections, there is going to be an explanation of the general structure of this configuration file, as well as each filter that can be specified to fulfill each gateway's responsibilities (section \ref{sec:microservices-gateway}).

\subsubsection*{General structure}

The general structure of the configuration file is going to be as follows: \\

\begin{lstlisting}[language=json,firstnumber=1]e
- path: path/to/my/resource
  envs: ['prd', 'stg', ... ]
  methods: ['get', 'post', 'put' ... ]
  endpointFilters:
    ...
  transport: 'http'
\end{lstlisting}

There will be one entry per resource on the gateway configuration file, and the following elements will specify each entry:

\begin{itemize}
    \item \texttt{path}: This is the URL that will be used to do the operation on the system, and it is also the identifier of the resource.
    \item \texttt{env}: The environment where the gateway is executed. Usually, there are three different environments considered in software engineering: local, staging, and production.
    \item \texttt{methods}: The HTTP method to be used: put, post, get, patch... those method corresponds to CRUD operations \cite{martin1983managing}.
    \item \texttt{endpointFilters}: Those filters are the responsibilities that the gateway will take. They are going to be defined in the following subsections.
    \item \texttt{transport}: The transport method used. Usually, it will correspond to HTTP, but other protocols like NATS can be specified.
\end{itemize}

\subsubsection*{Auditing filter}

The first filter specified, which is going to be contained under the \texttt{endpointFilters} specification, is going to be the filter used for the auditing responsibility of the gateway. The structure is as follows:\\

\begin{lstlisting}[language=json,firstnumber=1]e
  endpointFilters:
    - auditing: boolean
\end{lstlisting}

To keep this filter simple, there will be only a boolean flag that will determine if there has to be any kind of logging or auditing at the gateway level for the specified endpoint.

\subsubsection*{Identity filter}

This filter specifies if there has to be any authentication or authorization for consuming the endpoint resources. The structure of the filter is:\\

\begin{lstlisting}[language=json,firstnumber=1]e
  endpointFilters:
    - authentication: service-account | OAuth
    - authorization:
	    create: permission-identifier
	    read: permission-identifier
	    update: permission-identifier
	    delete: permission-identifier
\end{lstlisting}

First of all, the authentication section specifies how the user has to be identified on the system. In this example, two options are shown: service accounts and OAuth \cite{OAuth2}, but there can be many other methods and should be specified by the systems designers.\\

On the authorization side, the consideration has been to use Role-Based Access Control (RBAC) \cite{ferraiolo2003role}, which assigns one role to each user and specifies a set of permissions for each role. The specific CRUD operation can only be performed if the permission matches the one resolved for the user who tries to consume the resource.

\subsubsection*{Rate limiting filter}

Rate limiting filter defines the number of access that the gateway will accept on a certain period of time. The structure of the filter is the following: \\

\begin{lstlisting}[language=json,firstnumber=1]e
  endpointFilters:
    - httpMethods: [get, put, patch, delete] // http methods to limit
    - perSecond: {prd: 6, stg: 100, local: 10000}
    - perThirtyMinutes: 4000
\end{lstlisting}

There are three fields that the filter accepts; the first one is the HTTP methods that are considered for the rate-limiting, the second is the number of accesses that the gateway accepts in one second, and the third one is the number of access that it accepts in thirty minutes. For the last two properties, it can be specified the number of accesses or an object with three properties referring to the three possible environments where the application is deployed and a number for the rate-limiting on those.

\subsubsection*{Documentation filter}

This simple filter specifies if any documentation for the endpoint at the gateway level has to be produced. This makes sense because there might be private endpoints where the consumption is done only within the organization, so creating external documentation might not be desired. The structure of it is as follows:\\

\begin{lstlisting}[language=json,firstnumber=1]e
  endpointFilters:
    - documentation: boolean
\end{lstlisting}

\subsubsection*{Response fields trimming}

As the gateway should work as the proxy for all the microservices under a domain, we may want some fields not exposed outside of it, which is the purpose of the response field trimming. The structure is as follows:\\

\begin{lstlisting}[language=json,firstnumber=1]e
  endpointFilters:
    - responseTrimming: [field1, field2, ...]
\end{lstlisting}

The fields that are included in the array are not going to be forwarded to the endpoint consumers.

%% In this section describe the different high level errors and the standard that we are defining for them.
%% Check out the ones defined previously by Uber gateway API.

\subsection{Error schema generator} %%% OPTIONAL

%% Possible Open Source tool used to generate gateway routes schema definitions using a CLI.
%% If implemented, describe it here.

\subsection{Proof of concept using error rate tool}

%% The things that need to be covered here are:

%% - Planning (Using AirTable and estimation)
%% - Implementation on the gateway side. Explain that Unity currently has a logging tool enabled and we only had to log certain events gathering the requests.
%% - Terraform implementation. This part is where it can be explained how we use terraform and how to the problem cloud-agnostic.
%% - BigQuery setup.

\subsection{Error visualization dashboard}

%% Cover here the grafana dashboard implementation tool

%% - Cover the usage of the Grafana + BigQuery tool.
%% - Describe the different panels that had been developed.
%% - Persistance of the dashboard and deployment on every build.

\clearpage
\section{Evaluation}

%% The evaluation needs to be done for each of the artifacts separated:

\subsection{Evaluation of the error schema generator} 

%% Add the testing here if implemented.

\subsection{Evaluation of the proof of concept}

%% Here we can say two evaluation methods:
%% - Test script in order to inject mock data and see if it is storing correctly the data.
%% - Integration tests using jest on the gateway side.
%% - Using Proof of concept in order to test how the database was doing.
%% - Using staging and production in order to have different sources for the data.

\subsection{Evaluation of the visualization tool}

%% For the evaluation of the visualization tool we can explain the following:
%% - Using mock data in order to iterate over the dashboards.
%% - Using continuos reviews from the stakeholders. mention the API team and the gateway team.

\clearpage
\section{Conclusions}

%% This part is almost the last one to write, basically write some pages of conclussions, and offer future work

%% For future work basically it can be checked if there is something on the backlog.

%% Focus conclussion on an adoption analysis for the three types of organizations: startups, midsize organizations and large organizations.

\clearpage
\thesisbibliography

\bibliographystyle{plain}
\bibliography{bibliography}

%% Appendices
%% If you don't have appendices, remove \clearpage and \thesisappendix below.
\clearpage
\thesisappendix

\section{Appendix 1}

\clearpage
\section{Appendix 2}

\clearpage

\end{document}
